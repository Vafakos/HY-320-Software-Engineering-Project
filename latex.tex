\documentclass[a4paper,12pt]{article}
\usepackage[utf8]{inputenc}
\usepackage{amsmath,amsfonts,amssymb}
\usepackage{graphicx}
\usepackage{geometry}
\usepackage{hyperref}
\usepackage{alphabeta}
\usepackage{caption}
\usepackage{subcaption}
\usepackage{tikz}
\usepackage{pgf-umlcd}
\usepackage{hyperref}
\hypersetup{
    colorlinks=true,
    linkcolor=blue,
    urlcolor=blue
}


\geometry{a4paper, margin=1in}

\title{Εργασία Τεχνολογία Λογισμικού}
\author{Βαφάκος Χαράλαμπος Σωκράτης(Π2017015) \\  Φιλίππου Ίωνας(Π2017107)
\\ \href{https://github.com/Vafakos/HY-320-Software-Engineering-Project}{Αποθετήριο Github}}
\date{}

\begin{document}

\maketitle

\section{Εισαγωγή}
Στην εποχή της πληροφορίας, η εξόρυξη και ανάλυση δεδομένων έχει καταστεί αναγκαία σε πολλούς τομείς, όπως η επιχειρηματική ευφυΐα, η υγεία, και η επιστήμη. Αυτή η εργασία αποσκοπεί στην ανάπτυξη μιας web-based εφαρμογής για την εξόρυξη και ανάλυση δεδομένων, η οποία θα διευκολύνει τους χρήστες να εισάγουν, επεξεργάζονται και να αναλύουν δεδομένα μέσω φιλικών προς τον χρήστη διεπαφών και οπτικοποιήσεων.

Η εφαρμογή που αναπτύχθηκε χρησιμοποιεί το πλαίσιο ανάπτυξης Streamlit για τη δημιουργία μιας διαδραστικής ιστοσελίδας όπου οι χρήστες μπορούν να φορτώσουν δεδομένα, να εκτελέσουν αλγορίθμους μείωσης διάστασης όπως PCA και UMAP, και να εφαρμόσουν τεχνικές μηχανικής μάθησης για την ανάλυση των δεδομένων. Επιπλέον, η εφαρμογή παρέχει δυνατότητες επιλογής χαρακτηριστικών και κατηγοριοποίησης, επιτρέποντας τη σύγκριση των αποτελεσμάτων μέσω μετρικών όπως το accuracy, το f1-score και το roc-auc.

Η αναφορά αυτή περιγράφει τον σχεδιασμό και την ανάπτυξη της εφαρμογής, αναλύει τα αποτελέσματα που προέκυψαν από τη χρήση της και παρουσιάζει τα συμπεράσματα που προέκυψαν από την ανάλυση των δεδομένων.

\newpage

\section{Σχεδιασμός Εφαρμογής}
Η ανάπτυξη της εφαρμογής βασίστηκε σε μια modular αρχιτεκτονική, επιτρέποντας την εύκολη επέκταση και συντήρηση του κώδικα. Στο κέντρο της εφαρμογής βρίσκεται η δυνατότητα φόρτωσης δεδομένων από τον χρήστη, τα οποία στη συνέχεια επεξεργάζονται και αναλύονται σε διάφορα tabs που περιλαμβάνουν λειτουργίες όπως οπτικοποιήσεις, επιλογή χαρακτηριστικών και κατηγοριοποίηση.

Η εφαρμογή αποτελείται από τις εξής κύριες ενότητες:

\begin{itemize}
    \item \textbf{Φόρτωση Δεδομένων}: Ο χρήστης μπορεί να εισάγει δεδομένα σε μορφή πίνακα (CSV, Excel, κ.λπ.), τα οποία αναπαριστούν το σύνολο δεδομένων για ανάλυση.
    \item \textbf{Visualization Tab}: Σε αυτή την ενότητα, ο χρήστης μπορεί να οπτικοποιήσει τα δεδομένα του χρησιμοποιώντας αλγορίθμους μείωσης διάστασης όπως το PCA και το UMAP. Επιπλέον, παρέχονται διαγράμματα για διερευνητική ανάλυση δεδομένων (EDA).
    \item \textbf{Tabs Μηχανικής Μάθησης}: Αυτή η ενότητα περιλαμβάνει δύο καρτέλες:
        \begin{itemize}
            \item \textbf{Feature Selection Tab}: Ο χρήστης μπορεί να εκτελέσει έναν αλγόριθμο επιλογής χαρακτηριστικών, μειώνοντας το πλήθος των χαρακτηριστικών στο σύνολο δεδομένων.
            \item \textbf{Classification Tab}: Σε αυτή την καρτέλα, εκτελούνται αλγόριθμοι κατηγοριοποίησης, τόσο στο αρχικό όσο και στο μειωμένο σύνολο δεδομένων, με δυνατότητα σύγκρισης των αποτελεσμάτων.
        \end{itemize}
    \item \textbf{Info Tab}: Παρέχει πληροφορίες σχετικά με την εφαρμογή, τον τρόπο λειτουργίας της και την ομάδα ανάπτυξης.
\end{itemize}
\begin{figure}[ht!]
    \centering
    \includegraphics[width=0.75\linewidth]{UML.png}
    \caption{UML Diagram of the Application}
    \label{fig:uml-diagram}
\end{figure}
\section{Υλοποίηση}
\subsection{Φόρτωση Δεδομένων}
Η εφαρμογή ξεκινά με την καρτέλα "Upload File", όπου οι χρήστες μπορούν να ανεβάσουν σύνολα δεδομένων σε μορφές CSV, Excel ή TSV. Ο κώδικας ελέγχει το τύπο του αρχείου και φορτώνει τα δεδομένα στο περιβάλλον της εφαρμογής χρησιμοποιώντας τις βιβλιοθήκες \textit{pandas}. Επιπλέον, παρέχονται βασικές πληροφορίες σχετικά με το σύνολο δεδομένων, όπως ο αριθμός των δειγμάτων και των χαρακτηριστικών, καθώς και οι στήλες των χαρακτηριστικών και των ετικετών.

\subsection{Visualization Tab}
Στην καρτέλα "Visualization", οι χρήστες έχουν τη δυνατότητα να οπτικοποιήσουν τα δεδομένα χρησιμοποιώντας αλγορίθμους μείωσης διαστάσεων, όπως το PCA (Principal Component Analysis) και το UMAP (Uniform Manifold Approximation and Projection). Ο κώδικας επιτρέπει τη δημιουργία τόσο 2D όσο και 3D διαγραμμάτων, τα οποία βοηθούν στην ανάλυση της κατανομής των δεδομένων. Για την απεικόνιση των δεδομένων χρησιμοποιήθηκε η βιβλιοθήκη \textit{plotly}, η οποία παρέχει διαδραστικά γραφήματα.Επιπλέον, η εφαρμογή περιλαμβάνει επιλογές για την διεξαγωγή διερευνητικής ανάλυσης δεδομένων (EDA) μέσω διαγραμμάτων όπως Pairplot, Histogram και Box Plot, τα οποία βοηθούν στην κατανόηση της συμπεριφοράς των χαρακτηριστικών.

\subsection{Tabs Μηχανικής Μάθησης}

\subsubsection{Επιλογή Χαρακτηριστικών}
Στην καρτέλα "Feature Selection", χρησιμοποιείται η τεχνική SelectKBest για την επιλογή των πιο σημαντικών χαρακτηριστικών με βάση το κριτήριο \textit{chi-square}. Ο χρήστης μπορεί να επιλέξει τον αριθμό των χαρακτηριστικών που επιθυμεί να διατηρήσει, και τα επιλεγμένα χαρακτηριστικά εμφανίζονται σε ένα νέο σύνολο δεδομένων, το οποίο χρησιμοποιείται για περαιτέρω ανάλυση.

\subsubsection{Κατηγοριοποίηση Δεδομένων}
Η καρτέλα "Classification" επιτρέπει στους χρήστες να εφαρμόσουν αλγορίθμους κατηγοριοποίησης όπως K-Nearest Neighbors (KNN) και Random Forest, τόσο στο αρχικό όσο και στο μειωμένο σύνολο δεδομένων. Η εφαρμογή επιτρέπει την προσαρμογή παραμέτρων όπως ο αριθμός των γειτόνων (για το KNN) ή ο αριθμός των δέντρων (για το Random Forest), και υπολογίζει τις επιδόσεις των αλγορίθμων με βάση μετρικές όπως η ακρίβεια, το F1-Score και το ROC-AUC.

\subsubsection{Σύγκριση Αποτελεσμάτων}
Στην τελευταία καρτέλα, "Results and Comparison", παρουσιάζονται τα αποτελέσματα της κατηγοριοποίησης για το αρχικό και το μειωμένο σύνολο δεδομένων. Τα αποτελέσματα αυτά παρουσιάζονται μέσω διαδραστικών γραφημάτων που δείχνουν τις επιδόσεις των αλγορίθμων για κάθε σύνολο χαρακτηριστικών, επιτρέποντας στους χρήστες να συγκρίνουν και να αξιολογήσουν την αποτελεσματικότητα των τεχνικών επιλογής χαρακτηριστικών.

\section{Ανάλυση Αποτελεσμάτων}
Η ανάλυση των αποτελεσμάτων επικεντρώνεται στη σύγκριση της απόδοσης των αλγορίθμων ταξινόμησης πριν και μετά την εφαρμογή της επιλογής χαρακτηριστικών. Σκοπός αυτής της ανάλυσης είναι να κατανοηθεί πώς η μείωση των χαρακτηριστικών επηρεάζει τις μετρικές απόδοσης και να αξιολογηθεί η αποτελεσματικότητα των αλγορίθμων σε διαφορετικά σύνολα δεδομένων.

\subsection{Σύγκριση Αρχικού και Μειωμένου Συνόλου Δεδομένων}

\begin{itemize}
    \item \textbf{Ακρίβεια (Accuracy):} 
    Η ακρίβεια μετρά το ποσοστό των σωστών προβλέψεων σε σχέση με το σύνολο των προβλέψεων. Στα αποτελέσματα, συγκρίνονται οι τιμές ακρίβειας τόσο για το αρχικό όσο και για το μειωμένο σύνολο χαρακτηριστικών.
    Παρατηρείται εάν η μείωση των χαρακτηριστικών οδηγεί σε σημαντική μείωση ή βελτίωση της ακρίβειας. Εάν η ακρίβεια του μειωμένου συνόλου είναι συγκρίσιμη ή καλύτερη από την αρχική, αυτό δείχνει ότι η διαδικασία επιλογής χαρακτηριστικών ήταν επιτυχής.
    
    \item \textbf{F1-Score:}
    Το F1-Score συνδυάζει την ακρίβεια και την ανάκληση σε μία μετρική που λαμβάνει υπόψη την ισορροπία μεταξύ αυτών. Είναι ιδιαίτερα χρήσιμο όταν οι κλάσεις είναι ανισόρροπες.
    Συγκρίνονται οι τιμές F1-Score για τα δύο σύνολα δεδομένων, με στόχο να φανεί αν η επιλογή χαρακτηριστικών διατήρησε την ισορροπία μεταξύ της ανάκλησης και της ακρίβειας.
    
    \item \textbf{ROC-AUC:}
    Η καμπύλη ROC-AUC είναι ένα σημαντικό μέτρο για την απόδοση των ταξινομητών, ειδικά όταν οι κλάσεις δεν είναι ισορροπημένες. Ένα υψηλότερο ROC-AUC υποδηλώνει καλύτερο διαχωρισμό των κλάσεων από τον ταξινομητή.
    Η σύγκριση των τιμών ROC-AUC μεταξύ των δύο συνόλων δεδομένων δείχνει πόσο καλά ο ταξινομητής μπορεί να ξεχωρίσει τις κλάσεις μετά την επιλογή χαρακτηριστικών.
\end{itemize}

\subsection{Συμπεράσματα από τη Σύγκριση}

\begin{itemize}
    \item \textbf{Αποτελεσματικότητα της Επιλογής Χαρακτηριστικών:}
    Εάν οι μετρικές απόδοσης παραμένουν σταθερές ή βελτιώνονται μετά τη μείωση των χαρακτηριστικών, αυτό υποδηλώνει ότι η διαδικασία επιλογής έχει αφαιρέσει μη χρήσιμα ή θορυβώδη χαρακτηριστικά, αφήνοντας μόνο τα πιο σημαντικά για την πρόβλεψη.
    
    \item \textbf{Επιδόσεις Αλγορίθμων:}
    Οι αλγόριθμοι ταξινόμησης ενδέχεται να αντιδρούν διαφορετικά στη μείωση των χαρακτηριστικών. Για παράδειγμα, ο KNN μπορεί να επηρεαστεί περισσότερο από τον Random Forest λόγω της φύσης του αλγορίθμου που βασίζεται στη γειτνίαση. Τα αποτελέσματα θα αναλύσουν ποιος αλγόριθμος διαχειρίζεται καλύτερα τη μείωση των χαρακτηριστικών.
    
    \item \textbf{Συστάσεις για Βελτιώσεις:}
    Με βάση τα αποτελέσματα, μπορεί να προταθούν συστάσεις για βελτιώσεις. Για παράδειγμα, αν η μείωση των χαρακτηριστικών επηρεάζει αρνητικά τις μετρικές απόδοσης, ενδέχεται να χρειάζεται καλύτερη προεπεξεργασία των δεδομένων ή η χρήση πιο εξελιγμένων τεχνικών επιλογής χαρακτηριστικών.
\end{itemize}

\subsection{Συνολική Αξιολόγηση}

Η συνολική ανάλυση καταλήγει σε μια εκτίμηση του πώς τα αποτελέσματα του έργου πληρούν τους στόχους που τέθηκαν αρχικά. Αξιολογείται η αποδοτικότητα της υλοποίησης, οι επιλογές που έγιναν στους αλγορίθμους και τις τεχνικές, καθώς και η πρακτική σημασία των αποτελεσμάτων στην πραγματική εφαρμογή του συστήματος.


\section{Συμπεράσματα}
Η υλοποίηση της εφαρμογής παρέχει ένα ολοκληρωμένο εργαλείο για την εξόρυξη και ανάλυση δεδομένων. Η modular αρχιτεκτονική της επιτρέπει την εύκολη προσαρμογή και επέκταση, καθιστώντας την χρήσιμη σε ευρύ φάσμα αναλυτικών εφαρμογών. Μελλοντικές βελτιώσεις θα μπορούσαν να περιλαμβάνουν την προσθήκη περισσότερων αλγορίθμων ανάλυσης και την περαιτέρω βελτιστοποίηση της διεπαφής χρήστη.

\section{Κύκλος Ζωής Έκδοσης Λογισμικού}

Η ανάπτυξη του λογισμικού ακολουθεί την Agile μεθοδολογία, υποστηρίζοντας ταχεία ανάπτυξη και ευελιξία. Ο κύκλος ζωής περιλαμβάνει σχεδιασμό, ανάπτυξη, δοκιμές, έκδοση και συντήρηση. Στην κάθε φάση, εφαρμόζονται δοκιμές για εξασφάλιση ποιότητας και λειτουργικότητας, με τις εκδόσεις του λογισμικού να διανέμονται μέσω Docker για ευκολία και αξιοπιστία.

\subsection{Διαδικασίες Ανάπτυξης}

\begin{itemize}
    \item \textbf{Σχεδιασμός και Προγραμματισμός:} Ορισμός απαιτήσεων και ανάπτυξη λειτουργικοτήτων σε επαναληπτικές φάσεις (sprints).
    \item \textbf{Δοκιμές:} Εκτεταμένες δοκιμές (unit, integration, system tests) για εξασφάλιση ποιότητας.
    \item \textbf{Έκδοση:} Διανομή του λογισμικού μέσω Docker, εξασφαλίζοντας σταθερότητα και συμβατότητα.
    \item \textbf{Συντήρηση:} Συνεχείς ενημερώσεις και βελτιώσεις, ανταπόκριση σε νέες απαιτήσεις και επιδιορθώσεις σφαλμάτων.
\end{itemize}

Η επιλογή της Agile εξασφαλίζει ότι το λογισμικό μπορεί να προσαρμόζεται ταχύτατα σε αλλαγές, βελτιώνοντας την ικανότητα του έργου να ανταποκρίνεται στις εξελίξεις και τις ανάγκες των χρηστών.

\newpage

\section{Συνεισφορά Ομάδας}
\textbf{\textit{Βαφάκος Χαράλαμπος Σωκράτης:}}
\begin{itemize}
    \item Ανάπτυξη του κώδικα για τη φόρτωση αρχείων και την προεπεξεργασία δεδομένων.
    \item Δημιουργία της οπτικοποίησης με το PCA και ενσωμάτωση των ταξινομητών για το αρχικό σύνολο δεδομένων.
    \item Διαχείριση της επιλογής χαρακτηριστικών χρησιμοποιώντας το SelectKBest.
    \item Διαχείριση του αποθετηρίου GitHub και παρακολούθηση των αλλαγών.
\end{itemize}
\textbf{\textit{Φιλίππου Ίωνας:}}
\begin{itemize}
\item Συμμετοχή στην οργάνωση των δεδομένων και προσθήκη της στήλης με τις ετικέτες.
\item Δημιουργία της οπτικοποίησης με το UMAP και συμβολή στην επιλογή χαρακτηριστικών.
\item Εργασία στην κατηγοριοποίηση για το μειωμένο σύνολο δεδομένων και σύγκριση των αποτελεσμάτων.
\item Ανάληψη της δημιουργίας του Docker για την ανάπτυξη της εφαρμογής και βοήθεια στη σύνταξη της τελικής αναφοράς.
\item Σύνταξη πληροφοριών για την ανάπτυξη και τα μελλοντικά σχέδια στην καρτέλα Πληροφοριών.
\end{itemize}

\end{document}
